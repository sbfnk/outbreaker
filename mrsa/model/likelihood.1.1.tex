% Template for PLoS
% Version 1.0 January 2009
%
% To compile to pdf, run:
% latex plos.template
% bibtex plos.template
% latex plos.template
% latex plos.template
% dvipdf plos.template

\documentclass[10pt]{article}

% amsmath package, useful for mathematical formulas
\usepackage{amsmath}
% amssymb package, useful for mathematical symbols
\usepackage{amssymb}

% graphicx package, useful for including eps and pdf graphics
% include graphics with the command \includegraphics
\usepackage{graphicx}

% cite package, to clean up citations in the main text. Do not remove.
\usepackage{cite}

\usepackage{color} 

% Use doublespacing - comment out for single spacing
%\usepackage{setspace} 
%\doublespacing

\usepackage[frenchb,english]{babel}
\usepackage{bm}
\usepackage{cite}
\usepackage{amsthm}
\usepackage{amsfonts}
\usepackage{ccaption}
\usepackage{url}

% Text layout
\topmargin 0.0cm
\oddsidemargin 0.5cm
\evensidemargin 0.5cm
\textwidth 16cm 
\textheight 21cm

% Bold the 'Figure #' in the caption and separate it with a period
% Captions will be left justified
\usepackage[labelfont=bf,labelsep=period,justification=raggedright]{caption}

% Use the PLoS provided bibtex style
\bibliographystyle{plos2009}

% Remove brackets from numbering in List of References
\makeatletter
\renewcommand{\@biblabel}[1]{\quad#1.}
\makeatother


\begin{document}

% Title must be 150 words or less
% \begin{flushleft}
{\Large
\textbf{Likelihood}
}
% Insert Author names, affiliations and corresponding author email.
\\
\section*{Materials and Methods}

\subsection*{Observed data ($Y$)}

For each patient $i=1,\ldots,N$ admitted to one of the wards in the study period, we denote:
\begin{itemize}
	\item $w_i$ the ward where the patient is admitted ($1$ for adult ICU, $2$ for paediatric ICU)
	\item $k_i$ the number of times the patient is admitted (1 if no readmission)
	\item $A_i$ and $D_i$ vectors containing the times of admission and discharge from the ward
	\item $P_i$ and $N_i$ vectors containing the times of positive and negative swabs (positive defined as any of the samples taken is positive ; negative defined as all samples taken are negative).
	\item $p_i$ and $n_i$ the size of those vectors, ie the number of positive and negative swabs.
% 	\item $S_i = \{s_i^1, \ldots, s_i^{m_i}\}$ a set of $m_i$ genetic sequences of lengths $L_i = \{l_i^1, \ldots, l_i^{m_i}\}$ of MRSAs isolated in patient $i$ at times $T_i = \{t_i^1, \ldots, t_i^{m_i}\}$.
	\item $s_i$ a genetic sequence of the MRSA isolated in patient $i$ at time $t_i$.
% 	\item $\delta_{ab}$ and $\gamma_{ab}$ the number of transitions and transversions between isolates $a$ and $b$.
% 	\item $L_{ab}$ the length of the DNA sequence comparable between $a$ and $b$.
 	\item $\delta_{ij}$ and $\gamma_{ij}$ the number of transitions and transversions between isolates $i$ and $j$.
	\item $l_{ij}$ the length of the DNA sequence comparable between $i$ and $j$.

\end{itemize}

\subsection*{Augmented (unobserved) data ($Z$)}

For each patient $i$ admitted to one of the wards in the study period, we denote 
\begin{itemize}
	\item $C_i$ the colonisation time (we assume no supercolonisations)
	\item $E_i$ the time of end of colonisation.
\end{itemize}

\subsection*{Parameters ($\theta$)}

Parameters of the model are: 
\begin{itemize}
	\item $\beta$ a 2 by 2 matrix containing $\beta_{i,j}$, the person to person transmission rate from ward $j$ to ward $i$.
	\item $Sp$ the specificity of the testing, ie the probability of getting a negative test given uncolonized (assumed $100\%$).
	\item $Se$ the sensitivity of the testing, ie the probability of getting a positive test given colonized.
	\item $\pi$ the probability of being already colonized at first admission.
	\item $\mu$ and $\sigma$ the mean and standard deviation of the duration of colonization.
	\item $\nu_1$ and $\nu_2$ the rate of transitions ($A \leftrightarrow G$ and $C \leftrightarrow T$) and transversions (other changes) of the DNA sequences.
	\item $\tau$ the time to the most recent common ancestor of a pair of isolates for indirect ancestries (before the earliest collection date).
	\item $\alpha$ the probability of that a given isolate's ancestor belongs to the sample.
\end{itemize}

\subsection*{Statistical Model}

In the following, $\mathbf{1}_{\left\lbrace.\right\rbrace}$ denotes the indicator function, defined by $\mathbf{1}_{\left\lbrace X \right\rbrace}=1$ if $X$ is true, and $0$ otherwise.

The joint density of the observed data, the augmented data, and the model parameters is: 

\begin{equation} \label {JointDensity}
\begin{split}
P\left(\bm{Y},\bm{Z},\bm{\theta}\right) = P\left(\bm{Y}|\bm{Z}\right)P\left(\bm{Z}|\bm{\theta}\right)P\left(\bm{\theta}\right)\\ \nonumber
\end{split}
\end{equation}

\noindent where $P\left(\bm{Y}|\bm{Z}\right)$, $P\left(\bm{Z}|\bm{\theta}\right)$ 
and $P\left(\bm{\theta}\right)$ refer to the observation level, the transmission level
 and the prior level respectively.

%%%%%%%%

\subsubsection*{Observation level}

%%%%%%%%

The observation level ensures that the observed data are consistent with the augmented data:

\begin{eqnarray*}
P\left(\bm{Y}|\bm{Z}\right) =
\bm{\prod}_{i=1}^N % product on individuals
& \bm{\prod}_{j=1}^{p_i} % positive tests
\left(
\left( \mathbf{1}_{\left\lbrace P_i[j] < C_i \right\rbrace}+\mathbf{1}_{\left\lbrace P_i[j] > E_i \right\rbrace} \right) \times \left(1-Sp\right) % false positive
+ \mathbf{1}_{\left\lbrace C_i \leq P_i[j] \leq E_i \right\rbrace} \times Se % true positive
\right) \\ 
& \bm{\prod}_{k=1}^{n_i} % negative tests
\left(
\left( \mathbf{1}_{\left\lbrace N_i[k] < C_i \right\rbrace}+\mathbf{1}_{\left\lbrace N_i[k] > E_i \right\rbrace} \right) \times Sp % true negative
+ \mathbf{1}_{\left\lbrace C_i \leq N_i[k] \leq E_i \right\rbrace} \times \left(1-Se\right) % false negative
\right) \\ 
\end{eqnarray*}

The first line describes the positive tests, which can be either false positives (first term) or true positives (second term). 
The second line describes the negative tests, which can be either true negatives (first term) or false negatives (second term). 

%%%%%%%%

\subsubsection*{Transmission level} 

%%%%%%%%

\begin{eqnarray*}
P\left(\bm{Z}|\bm{\theta}\right) = 
\bm{\prod}_{i=1}^N % product on individuals
\left(
\pi \times \mathbf{1}_{\left\lbrace C_i < A_i[1] \right\rbrace}   % colonized before first admission
+ \left( 1-\pi \right) \times \bm{\sum}_{l=1}^{k_i} % uncolonized at first admission
	\left(
			\mathbf{1}_{\left\lbrace A_i[l] < C_i < D_i[l] \right\rbrace} \times exp\left(-\int_{u=A_i[l]}^{C_i} sum_{j=1}^2 \beta_{w_i,j} I^j\left(u\right)du - \beta^*\left(C_i-A_i[l]\right)\right) \times \left(sum_{j=1}^2 \beta_{w_i,j}f_{w_i,j}\left(\right) I^j\left(C_i\right) +\beta^*f^*\left(\right) \right) % colonized during the l^th stay in the ward			
			+ \mathbf{1}_{\left\lbrace C_i > D_i[l] \right\rbrace} \times exp\left(-\int_{u=A_i[l]}^{D_i[l]} sum_{j=1}^2 \beta_{w_i,j} I^j\left(u\right)du - \beta^*\left(D_i[l]-A_i[l]\right)\right) % uncolonized at the end of the l^th stay in the ward
	\right)
\right)
\times pdfGamma_{\mu,\sigma}\left(E_i-C_i\right)
\end{eqnarray*}
\noindent where $pdfGamma_{\mu,\sigma}$ is the probability density function of a Gamma distribution with mean $\mu$ and standard deviation $\sigma$ (we assume that the duration of colonisation is Gamma distributed). 


%%%%%%%%

\subsubsection*{Genetic likelihood}

%%%%%%%%

The probability $f_{ij}$ of observing sequences $i$ and $j$ given that patient $i$ infected patient $j$ is:
\begin{eqnarray*}
f_{ij} & = & \alpha \times \left( \mathcal{P}(\delta_{ij} | \nu_1 (t_j-t_i) l_{ij}) 
  + \mathcal{P}(\gamma_{ij} | \nu_2 (t_j-t_i) l_{ij}) \right) + \\
& & (1-\alpha) \times \left( \mathcal{P}(\delta_{ij} | \nu_1 (t_j-t_i + 2\tau) l_{ij}) + \mathcal{P}(\gamma_{ij} | \nu_2 (t_j-t_i+ 2\tau) l_{ij}) \right)
\end{eqnarray*}
where $\mathcal{P}$ is the probability mass function of a Poisson distribution.

%%%%%%%%


%%%%%%%%

\subsubsection*{Prior level}

%%%%%%%%

For all model parameters, independent prior distributions (TO SPECIFY) were chosen. 
  
\subsection*{Parameter Estimation}

A Markov chain Monte Carlo (MCMC) method was used to sample the joint
posterior distribution $P\left(\bm{Y},\bm{Z},\bm{\theta}\right)$.

\end{document}

